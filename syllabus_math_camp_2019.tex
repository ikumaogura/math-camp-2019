\documentclass[pdflatex, letterpaper, 12pt]{scrartcl}

\usepackage[truedimen, left=1truein, right=1truein, top=1truein, bottom=1truein]{geometry}
\usepackage{amsmath, amssymb}
\usepackage{here}
\usepackage{newtxtext, newtxmath}
\usepackage[bookmarks, colorlinks, breaklinks, linkcolor=blue, citecolor=blue, urlcolor=red, pdftitle={}, pdfauthor={}, pdfkeywords={}]{hyperref}


\title{Government Ph.D. Math Camp}
\author{Ikuma Ogura \\ Elena Wicker}
\date{August 19-23, 2019}

\begin{document}

\maketitle

\section*{Logistics}

\begin{itemize}
\item Contact information
 \begin{itemize}
 \item Ikuma: \href{mailto:io85@georgetown.edu}{io85@georgetown.edu}
 \item Elena: \href{mailto:ecw73@georgetown.edu}{ecw73@georgetown.edu}
 \end{itemize}
\item Course website: \href{https://blogs.commons.georgetown.edu/government-math-camp/}{https://blogs.commons.georgetown.edu/government-math-camp/}
\item When: August 19, 1:00 pm - 5:00pm; August 20 - 23, 9:00 am - 5:00 pm
\item Where: ICC 119
\end{itemize}

\section*{Course Description}

This is a one-week intensive course designed to introduce mathematial and programming skills that political science Ph.D. students will use in their quantitative methods and formal modeling courses. Topics covered in this course include (but not limited to): log and exponential functions, limits, derivatives, integrals, matrix algebra, probability theory, and introduction to \texttt{R} statistical software. Although attendance is not mandatory, it is highly encouraged especially if you do not feel confident in mathematics. This course is not graded, and answers to the homework assignments will be reviewed in class.   

\section*{Textbook \& References}

The electronic version of the following book is available through the \href{https://www.library.georgetown.edu/}{Georgetown University library website} and will be used as the textbook of this course.
\begin{quote}
Moore, Will H. and David A. Siegel. 2013. \emph{A Mathematics Course for Political and Social Research.} Princeton University Press.
\end{quote}
Since we go through the materials very fast, we recommend reading the relevant sections of this book before coming to the class.

\medskip
Additional books which will be helpful for preparation and review of the course materials:
\begin{itemize}
\item Gill, Jeff. 2006. \emph{Essential Mathematics for Political and Social Research}. Cambridge University Press.
\item Imai, Kosuke. 2018. \emph{Quantitative Social Science: An Introduction}. Princeton University Press.
\end{itemize}
Gill (2006) is a mathematics textbook for social science students similar to Moore and Siegel (2013). Imai (2018) is an introductory textbook on \texttt{R} language and probability theory. 

\section*{Tentative Daily Schedule (Tue - Fri)}

\begin{table}[H]
\begin{tabular}{l l }
9:00 - 10:00 am & Review Homework \\
10:10 - 12:00 pm & Lecture 1 \\
12:00 - 1:00 pm & Lunch break \\
1:00 - 2:00 pm & \texttt{R} Lab \\
2:10 - 4:00 pm & Lecture 2 \\
4:00 - 5:00 pm & Office hours \\
\end{tabular}
\end{table}

\section*{Tentative Lecture Topics}

\begin{description}
\item[August 19 (Mon): Basics] \ 
 \begin{itemize}
 \item Notations
 \item Functions, equations, inequalities
 \item Graphing functions
 \item Logarithms and exponents
 \item \textbf{Moore and Siegel, Chs 1 - 3.2.}
 \item \textbf{\textit{Note: International students with F-1/J-1 visa must attend the International student orientation in the morning. Therefore, we start the class at 1:00 pm and plan to end at 5:00 pm.}}
 \end{itemize}
\item[August 20 (Tue): Calculus 1] \ 
 \begin{itemize}
 \item Limits
 \item Univariate derivative
 \item Unconstrained optimization
 \item \textbf{Moore and Siegel, Chs 4 - 6 \& 8.} 
 \end{itemize}
\item[August 21 (Wed): Calculus 2] \
 \begin{itemize}
 \item Multivariate derivative
 \item Integral
 \item \textbf{Moore and Siegel, Ch 7.}
 \end{itemize}
\item[August 22 (Thu): Matrix Algebra] \ 
 \begin{itemize}
 \item Vector/Matrix calculation
 \item Geometric meanings of vector/matrix operations
 \item \textbf{Moore and Siegel, Chs 12-13.}
 \end{itemize}
\item[August 23 (Fri): Probability Theory] \
 \begin{itemize}
 \item Probability
 \item Random variable
 \item Discrete probability distribution
 \item Continuous probability distribution 
 \item \textbf{Moore and Siegel, Chs 9-11.}
 \end{itemize}
\end{description}



\end{document}



